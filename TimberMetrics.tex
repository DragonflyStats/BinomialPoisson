\section{Timber metrics}

The amount of standing timber that a forest contains is determined from:

\begin{itemize}
\item \textbf{Age class (Size class)}: This is a misnomer as in German it is Wuchsklass (Growth class) and properly should be the size of the tree (Size class), which may be limited by shade over it and not necessarily the biological and thus physiological age of the tree. This distinction is important if tree growth over time is expected by an owner or forester to produce timber, as a small old tree will grow differently from a small young tree. Commonly these age classes are split into: Seedling, Sapling, Pole, Mature Tree (subdivided into Weak wood, Middle wood and Strong wood stages), Old / Scenescant Tree. Sometimes it is called size class or a cohort. There are differences between countries and forests.
\item \textbf{Basal area }:  defines the area of a given section of land that is occupied by the cross-section of tree trunks and stems at their base
\item \textbf{Diameter at breast height (DBH) }:  measurement of a tree's girth standardized with different countries having different standards they are often at 1.3 meters (about 4.5 feet) above the ground
\item \textbf{Form factor }:  the shape of the tree, based on recorded trees and commonly then given for calculating tree volumes for a given species. It is usually related to DBH or age class. It is distinct from taper.[3] So it can be conical or paraboloid for example.
\item \textbf{Girard form class }:  an expression of tree taper calculated as the ratio of diameter inside the bark at 16 feet above ground to that outside bark at DBH, primary expression of tree form used in the United States
\item \textbf{Quadratic mean diameter }:  diameter of the tree that coordinates to the stand's basal area
\item \textbf{Site index }:  a species specific measure of site productivity and management options, reported as the height of dominant and co-dominant trees (site trees)in a stand at a base age such as 25, 50 and 100 years
\item \textbf{Tree taper}: the degree to which a tree's stem or bole decreases in diameter as a function of height above ground. So it can be sharp or gradual.
\end{itemize}
\end{document}
