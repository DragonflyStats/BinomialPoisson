\subsection*{First thinning of conifer forests}

What is thinning?

\begin{itemize}
\item When a new farm forest is established the trees are planted at a stocking rate of at least 2,500 trees to the hectare. After a number of years they begin to compete with each other as they grow. If the plantation is not thinned it is likely that by the time the crop is ready for clearfelling the number of trees per hectare would be about 1,400, the remainder having died off due to natural competition in the crop. The average tree volume would be about 0.4 m3.
\item 
However, if a percentage of the trees are removed at various stages during the life of the crop the remaining trees have more growing space, resulting in fewer trees but of greater quality and size. In a plantation that has been thinned throughout its life there should be about 500 trees per hectare remaining at the time of clearfell with each tree having a volume of 0.8 m3. This is twice the size of trees in an unthinned plantation. This results in a more valuable crop as larger trees command a much higher price as they are used to produce higher value products.
\item 
If properly carried out, thinning optimises the return from your forest crop, provides periodic returns as the crop matures and improves the biodiversity of the forest. Not thinning will result in a larger number of smaller sized trees, with a likely reduction in crop value.
\end{itemize}

\end{document}
