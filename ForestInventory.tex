\section{Forest inventory}
Forest inventory is the systematic collection of data and forest information for assessment or analysis. 
An estimate of the value and possible uses of timber is an important part of the broader information required 
to sustain ecosystems.[1] When taking forest inventory the following are important things to measure and 
note: species, diameter at breast height (DBH), height, site quality, age, and defects. 

From the data collected one can calculate the number of trees per acre, the basal area, the volume of trees in an area, 
and the value of the timber. Inventories can be done for other reasons than just calculating the value. 

A forest can be cruised to visually assess timber and determine potential fire hazards and the risk of fire. 
The results of this type of inventory can be used in preventive actions and also awareness. 

Wildlife surveys can be undertaken in conjunction with timber inventory to determine the number and type of wildlife within a forest. 
The aim of the statistical forest inventory is to provide comprehensive information about the state and dynamics of forests for strategic and management planning. 
Merely looking at the forest for assessment is called taxation. 

\end{document}
