\documentclass{beamer}

\usepackage{amsmath}
\usepackage{framed}

\begin{document}
%------------------------------------------------------------------------------------ %
\begin{frame}
\bigskip
{
\Huge
\[ \mbox{Probability Distributions with \texttt{R}} \]
\huge
\[ \mbox{Poisson Random Variables} \]
}
{
\huge
\[ \mbox{www.Stats-Lab.com} \]
\[ \mbox{Twitter: @StatsLabDublin} \]
}
\end{frame}
\begin{frame}
\frametitle{Poisson Random Variables}
\noindent\textbf{Question:}\\
Suppose that random variable X follows a Poisson distribution with rate parameter \texttt{\textbf{L}}. \\
If we increase the value of \texttt{\textbf{L}}, which of the following is true?


\bigskip
\noindent\textbf{Options:}
\begin{enumerate}
\item The spread increases but the center remains unchanged.
\item Both the spread and the center increase.
\item The center increases but the spread decreases.
\item The spread increases but the center decreases.
\end{enumerate}
\end{frame}

\begin{frame}
\frametitle{Poisson Random Variables}
\vspace{0.5cm}
\Large
\noindent \textbf{Comments:}
\begin{itemize}
\item center - i.e. the measures of centrality, such as mean and median.
\item spread - i.e. measures of disperion, such as variance and range.
\end{itemize} 
\end{frame}
%----------------------------------------------------------- %
\begin{frame}[fragile] 
\textbf{Exercise}
\begin{itemize}
\item Generate 100 random numbers from the Poisson distribution - specifying a value for \texttt{lambda} (i.e. what the rate parameter is called when using \texttt{R}) .
\item Compute the mean and variance for this set of numbers.
\item Repeat the process a few times, each time increasing the value of lambda.
\end{itemize}
\begin{framed}
\begin{verbatim}
#generate three data sets
X1 <- rpois(100, lambda= 4)
X2 <- rpois(100, lambda= 8)
X3 <- rpois(100, lambda= 18)
\end{verbatim}
\end{framed}
\end{frame}
%----------------------------------------------- %
\begin{frame}[fragile]
\frametitle{Poisson Random Variables}
\begin{framed}
\begin{verbatim}
#Now get the mean and variance for each data set
mean(X1);var(X1)
mean(X2);var(X2)
mean(X3);var(X3)
\end{verbatim}
\end{framed}
\end{frame}
\begin{frame}
END
\end{frame}
\end{document}