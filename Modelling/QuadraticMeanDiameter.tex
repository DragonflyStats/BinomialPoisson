\section{Quadratic Mean Diameter}

In forestry, quadratic mean diameter or QMD is a measure of central tendency which is considered more appropriate than 
arithmetic mean for characterizing the group of trees which have been measured. 
For n trees, QMD is calculated using the quadratic mean formula:

\[ {\displaystyle {\sqrt {\frac {\sum {D_{i}}^{2}}{n}}}}\]

where ${\displaystyle {D_{i}}}$ is the diameter at breast height of the ith tree. 

Compared to the arithmetic mean, QMD assigns greater weight to larger trees – QMD is always greater than or equal to 
arithmetic mean for a given set of trees. QMD can be used in timber cruises to estimate the standing volume of timber 
in a forest, because it has the practical advantage of being directly related to basal area, which in turn is directly
related to volume.[1] QMD can also be calculated as:

\[ {\displaystyle {\sqrt {\frac {BA}{k*n}}}}\]

where BA is stand basal area, n is the number of trees, and k is a constant based on measurement units - for BA in square feeet and
DBH in inches, k=0.005454; for BA in m2 and DBH in cm, k=0.0000785. 

\end{document}
