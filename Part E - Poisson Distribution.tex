
\documentclass[]{article}
\voffset=-1.5cm
\oddsidemargin=0.0cm
\textwidth = 470pt

% http://www.strath.ac.uk/aer/materials/5furtherquantitativeresearchdesignandanalysis/unit6/whatislogisticregression/

% http://www.medcalc.org/manual/logistic_regression.php


\usepackage{amsmath}
\usepackage{graphicx}
\usepackage{amssymb}
\usepackage{framed}
\begin{document}

\tableofcontents

\newpage
%========================================================= %
\Large

\subsection{Poisson probability distribution}

A discrete random variable that is often used is one which estimates the number of occurrences  over a specified time period or space.

(remark : a specified space can be a specified length , a specified area, or a specified volume.)

If the following two properties are satisfied, the number of occurrences is a random variable described by the Poisson probability distribution
%======================================================================================= %

\textbf{Properties}\\
1)      The probability of an occurrence is the same for any two intervals of equal length.\\
2)      The occurrence or non-occurrence in any interval is independent of the occurrence or non-occurrence in any other interval.\\

%======================================================================================= %


\section{Poisson Formulae}
	Given that there is on average 2 occurrences per hour, what is the probability of no occurrences in the next hour? \\ i.e. Compute $P(X=0)$ given that $m=2$
	\Large
	\[ P(X = 0)=\frac{2^0 e^{-2}}{0!} \]
	\normalsize
	\begin{itemize}
		\item $2^0$ = 1
		\item $0!$ = 1
	\end{itemize}
	The equation reduces to
	\[ P(X = 0)=e^{-2} = 0.1353\]

%======================================================================================= %	
\newpage
\large
\section*{The Poisson Distribution}
\textit{(Part E - Probability Distributions)}
\begin{itemize}
\item The Poisson Distribution is a statistical distribution showing the frequency probability of specific events when the average probability of a single occurrence is known. 

\item The Poisson distribution is a discrete probability distribution.

\item Application of the Poisson distribution enables managers to introduce optimal scheduling systems. 

\item For example, if the average number of people that rent movies on a friday night at a single video store location is 400,  a Poisson distribution can answer such questions as, "What is the probability that more than 600 people will rent movies?".



\item One of the most famous historical practical uses of the Poisson distribution was estimating the annual number of Prussian cavalry soldiers killed due to horse-kicks. 

\item Other modern examples include estimating the number of car crashes in a city of a given size; 
in physiology, this distribution is often used to calculate the probabilistic frequencies of different types of neurotransmitter secretions. 
\end{itemize}
%---------------------------------------------------------------------------%

%\frametitle{Poisson Random Variables}
\noindent\textbf{Question:}\\
Suppose that random variable X follows a Poisson distribution with rate parameter \texttt{\textbf{L}}. \\
If we increase the value of \texttt{\textbf{L}}, which of the following is true?


\bigskip
\noindent\textbf{Options:}
\begin{enumerate}
	\item The spread increases but the center remains unchanged.
	\item Both the spread and the center increase.
	\item The center increases but the spread decreases.
	\item The spread increases but the center decreases.
\end{enumerate}



%\frametitle{Poisson Random Variables}
%\vspace{0.5cm}
\Large
\noindent \textbf{Comments:}
\begin{itemize}
	\item center - i.e. the measures of centrality, such as mean and median.
	\item spread - i.e. measures of disperion, such as variance and range.
\end{itemize} 

%----------------------------------------------------------- %

\textbf{Exercise}
\begin{itemize}
	\item Generate 100 random numbers from the Poisson distribution - specifying a value for \texttt{lambda} (i.e. what the rate parameter is called when using \texttt{R}) .
	\item Compute the mean and variance for this set of numbers.
	\item Repeat the process a few times, each time increasing the value of lambda.
\end{itemize}
\begin{framed}
	\begin{verbatim}
	#generate three data sets
	X1 <- rpois(100, lambda= 4)
	X2 <- rpois(100, lambda= 8)
	X3 <- rpois(100, lambda= 18)
	\end{verbatim}
\end{framed}

%----------------------------------------------- %

%\frametitle{Poisson Random Variables}
\begin{framed}
	\begin{verbatim}
	#Now get the mean and variance for each data set
	mean(X1);var(X1)
	mean(X2);var(X2)
	mean(X3);var(X3)
	\end{verbatim}
\end{framed}
%============================================================================================================================= %

\section{Notation for Poisson Distribution}
A discrete random variable $X$ is said to follow a Poisson distribution with parameter $m$, written $X \sim \mbox{Po}(m)$, if it has probability distribution


\[ P(X=k) = e^{-m} {m^k \over k!} \]

where
\begin{itemize}
	\item $k = 0, 1, 2, \ldots$
	\item $m > 0$.
\end{itemize}

The Poisson probability function is given by



\begin{itemize} 
	\item	f(x) the probability of x occurrences in an interval. 
	\item	$\lambda$ is the expected value of the mean number of occurrences in any interval. (We often call this the Poisson mean)
	\item	e=2.71828284
\end{itemize} 


%============================================================================================= %

\subsection{Binomial Distribution: Worked Example}
Suppose a gambler is playing a simple coin flip game. 
The gambler does not know that the coin has been tampered with such that the probability of a Head is 47\%.

Suppose the gamble plays this coin flip game nine times. 
What is the probability that he wins precisely 3 times.
%----------------------------------------------%

	\section{Characteristics of a Poisson Experiment}
	A Poisson experiment is a statistical experiment that has the following properties:
	\begin{itemize}
		\item The experiment results in outcomes that can be classified as successes or failures.
		\item The average number of successes (m) that occurs in a specified region is known.
		\item The probability that a success will occur is proportional to the size of the \textbf{\emph{region}}.
		\item The probability that a success will occur in an extremely small region is virtually zero.
		\item The \texttt{pois} family of functions are used to compute probabilities and quantiles.
	\end{itemize}
	Note that the specified region could take many forms. For instance, it could be a length, an area, a volume, a period of time, etc.
	
	\subsection{The Poisson Random Variable}
	\begin{itemize}
		\item A Poisson random variable is the number of successes that result from a Poisson experiment.
		\item The probability distribution of a Poisson random variable is called a Poisson distribution.
		\item This distribution describes the number of occurrences in a unit period (or space)
		\item The expected number of occurrences is $m$.
		\item \texttt{R} refers to the mean number of occurrences as \texttt{lambda} rather than \texttt{m}. 
	\end{itemize}
	
	The probability that there will be $k$ occurrences in a unit time period is denoted $P(X=k)$, and is computed as below. Remark: This is known as the probability density function. The corresponding \texttt{R} command is \texttt{dpois()}.
	\Large
	\[ P(X = k)=\frac{m^k e^{-m}}{k!} \]
	
	
	

	\subsection{Example}
	What is the probability of one occurrences in the next hour? \\ i.e. Compute $P(X=1)$ given that $m=2$
	\Large
	\[ P(X = 1)=\frac{2^1 e^{-2}}{1!} \]
	\normalsize
	\begin{itemize}
		\item $2^1$ = 2
		\item $1!$ = 1
	\end{itemize}
	The equation reduces to
	\[ P(X = 1) = 2 \times e^{-2} = 0.2706\]

\subsection{Poisson probability distribution}

A discrete random variable that is often used is one which estimates the number of occurrences  over a specified time period or space.

(remark : a specified space can be a specified length , a specified area, or a specified volume.)

If the following two properties are satisfied, the number of occurrences is a random variable described by the Poisson probability distribution

\textbf{Properties}
\begin{enumerate}
\item      The probability of an occurrence is the same for any two intervals of equal length.\\
\item   The occurrence or non-occurrence in any interval is independent of the occurrence or non-occurrence in any other interval.
\end{enumerate}

The Poisson probability function is given by



\begin{itemize} 
	\item	f(x) the probability of x occurrences in an interval. 
	\item	$\lambda$ is the expected value of the mean number of occurrences in any interval. (We often call this the Poisson mean)
	\item	e=2.71828284
\end{itemize} 


\subsection{Poisson Approximation of the Binomial Probability Distribution}

The Poisson distribution can be used  as an approximation of the binomial probability distribution when p, the probability of success is small and n, the number of trials is large.
We set   (other notation  )  and use the Poisson tables. 

As a rule of thumb, the approximation will be good wherever both  and  


%======================================================================================= %

%---------------------------------------------------------------%
\section{Poisson Distribution (Example)}
	\begin{itemize}
		
		\item Suppose that electricity power failures occur according to a Poisson distribution
		with an average of 2 outages every twenty weeks. \item Calculate the probability that there will
		not be more than one power outage during a particular week.
	\end{itemize}
	
	\textbf{Solution:}
	
	\begin{itemize}
		\item The average number of failures per week is: $m = 2/20 = 0.10$
		
		\item ``Not more than one  power outage" means we need to compute and add the probabilities for ``0 outages" plus ``1 outage".
	\end{itemize}
	

	
	Recall: \[P(X = k) = e^{-m}\frac{m^k}{k!}\]
	
	
	\begin{itemize}
		
		\item $P(X = 0)$
		
		\[P(X = 0) = e^{-0.10}\frac{0.10^0}{0!} = e^{-0.10} = 0.9048\]
		
		
		\item $P(X = 1)$
		
		\[P(X = 1) = e^{-0.10}\frac{0.10^1}{1!} = e^{-0.10}\times 0.1 = 0.0905\]
		
		\item $P(X \leq 1)$
		
		\[P(X \leq 1) = P(X = 0) + P(X = 1) = 0.9048 + 0.0905 = 0.995\]
		
	\end{itemize}
	
	\begin{itemize}
		\item Probability Density Function $P(X = k)$
		\begin{itemize}
			\item For a given poisson mean $m$, which in \texttt{R} is specified as \texttt{lambda} 
			\item \texttt{dpois(k,lambda = ...)} 
		\end{itemize}
		\item Cumulative Density Function $P(X \leq k)$
		\begin{itemize}
			\item \texttt{ppois(k,lambda = ...)}
		\end{itemize}
	\end{itemize}
	

	%\frametitle{Implementation using \texttt{R}}
	From before: $P(X = 0)$ given than the mean number of occurrences is 2.
	
	\begin{verbatim}
	> dpois(0,lambda=2)
	[1] 0.1353353
	> dpois(1,lambda=2)
	[1] 0.2706706
	> dpois(2,lambda=2)
	[1] 0.2706706
	\end{verbatim}

	%\frametitle{Implementation using \texttt{R}}
	Compute the cumulative distribution functions for the values $k=\{0,1,2\}$, given that the mean number of occurrences is 2
	
	\begin{verbatim}
	> ppois(0,lambda=2)
	[1] 0.1353353
	> ppois(1,lambda=2)
	[1] 0.4060058
	> ppois(2,lambda=2)
	[1] 0.6766764
	\end{verbatim}
	

	%\frametitle{Poisson Approximation of the Binomial}
	\begin{itemize}
		\item The Poisson distribution can sometimes be used to approximate the binomial distribution
		\item When the number of observations n is large, and the success probability p is small, the $\mbox{Bin}(n,p)$ distribution approaches the Poisson distribution with the parameter given by $m = np$.
		\item This is useful since the computations involved in calculating binomial probabilities are greatly reduced.
		\item As a rule of thumb, n should be greater than 50 with p very small, such that $np$ should be less than 5.
		\item If the value of $p$ is very high, the definition of what constitutes a ``success" or ``failure" can be switched.
	\end{itemize}
\subsection{Poisson Approximation: Example}
	
	Suppose we sample 1000 items from a production line that is producing, on average, 0.1\% defective components.\\
	
	
	\bigskip
	
	Using the binomial distribution, the probability of exactly 3 defective items in our sample is
	
	\[P(X=3) = ^{1000}C_3 \times (0.001)^3 \times 0.999^{997} \]
	

	%\frametitle{Poisson Approximation: Example}
	Lets compute each of the component terms individually.
	
	
	\begin{itemize}
		\item $^{1000}C_3$
		
		\[ ^{1000}C_3 = \frac{1000 \times 999 \times 998}{3 \times 2 \times 1} =
		166,167,000 \]
		
		\item $0.001^3$
		
		\[0.001^3 = 0.000000001 \]
		
		
		\item $0.999^{997}$
		
		\[0.999^{997} = 0.36880 \]
		
	\end{itemize}
	Multiply these three values to compute the binomial probability \[P(X=3) = 0.06128 \]
	
%\frametitle{Poisson Approximation: Example}
	\begin{itemize}
		\item Lets use the Poisson distribution to approximate a solution.
		
		\item First check that $n \geq 50$ and $np <5$ (Yes to both).
		
		\item We choose as our parameter value $m = np = 0.001 \times 1000  = 1$
		
		\[P(X=3) = e^{-1}\frac{1^3}{3!} = \frac{e^{-1}}{6} = \frac{0.36787}{6} =  0.06131\]
		\item Compare this answer with the Binomial probability \\ $P(X=3) = 0.06128$.
		\item Very good approximation, with much less computation effort.
	\end{itemize}

	\begin{verbatim}
	> # Poisson Mean m = 1000 * 0.001 = 1
	> dbinom(3,size=1000,prob=0.001)
	[1] 0.06128251
	>
	> dpois(3,lambda=1)
	[1] 0.06131324
	\end{verbatim}
	


%---------------------------------------------------------------------------%
	%\frametitle{Continuous Random Variables}
	
	\begin{itemize}
		\item Probability Density Function
		\item Cumulative Density Function
	\end{itemize}
	
	
	If X is a continuous random variable then we can say that the probability of obtaining a \textbf{precise} value $x$ is infinitely small, i.e. close to zero.
	
	\[P(X=x) \approx 0 \]
	
	Consequently, for continuous random variables (only),  $P(X \leq x)$ and $P(X < x)$ can be used interchangeably.
	
	\[P(X \leq x) \approx P(X < x) \]
	
	
	%\frametitle{Continuous Uniform Distribution}
	A random variable X is called a continuous uniform random variable over the interval $(a,b)$ if it's probability density function is given by
	
	\[ f_{X}(x)  =  { 1 \over b-a}   \hspace{2cm}  \mbox{ when } a \leq x \leq b\]
	
	The corresponding cumulative density function is
	
	\[ F_x(x) = { x-a \over b-a}   \hspace{2cm}  \mbox{ when } a \leq x \leq b\]
	
	
	The mean of the continuous uniform distribution is
	
	\[ E(X) = {a+b \over 2}\]
	
	\[ V(X) = {(b-a)^2\over12}\]

	%\frametitle{The Memoryless property}
	The most interesting property of the exponential distribution is the \textbf{\emph{memoryless}} property. By this , we mean that if  the lifetime of a component is exponentially distributed, then an item which has been in use for some time is a good as a brand new item with regards to the likelihood of failure.
	
	The exponential distribution is the only distribution that has this property.

	%\frametitle{Random Variables}
	A pair of dice is thrown. Let X denote the minimum of the two numbers which occur.
	Find the distributions and expected value of X.

%\frametitle{Random Variables}
	A fair coin is tossed four times.
	Let X denote the longest string of heads.
	Find the distribution and expectation of X.
	
%-------------------------------------------------------------%
{\Large%\frametitle{Random Variables}
	A fair coin is tossed until a head or five tails occurs.
	Find the expected number E of tosses of the coin.}
%-------------------------------------------------------------%
{\Large%\frametitle{Random Variables}A coin is weighted so that P(H) = 0.75 and P(T ) = 0.25
	
	The coin is tossed three times. Let X denote the number of
	heads that appear.
	\begin{itemize}
		\item (a) Find the distribution f of X.
		\item (b) Find the expectation E(X).
	\end{itemize}

	\begin{itemize}
		\item Now consider an experiment with only two outcomes. Independent repeated trials of such an experiment are
		called Bernoulli trials, named after the Swiss mathematician Jacob Bernoulli (1654–1705). \item The term \textbf{\emph{independent
				trials}} means that the outcome of any trial does not depend on the previous outcomes (such as tossing a coin).
		\item We will call one of the outcomes the ``success" and the other outcome the ``failure".
	\end{itemize}
}

%-------------------------------------------------------------%
{\Large
	\begin{itemize}
		\item
		Let $p$ denote the probability of success in a Bernoulli trial, and so $q = 1 - p$ is the probability of failure.
		A binomial experiment consists of a fixed number of Bernoulli trials. \item A binomial experiment with $n$ trials and
		probability $p$ of success will be denoted by
		\[B(n, p)\]
	\end{itemize}
}
%-------------------------------------------------------------%

%---------------------------------------------------------------------------%
{\Large
	%\frametitle{Probability Mass Function}
	\begin{itemize} \item a probability mass function (pmf) is a function that gives the probability that a discrete random variable is exactly equal to some value. \item The probability mass function is often the primary means of defining a discrete probability distribution \end{itemize}
}
\subsection{Poisson }

M=15
(1/2 hour or 30 minutes)

5 minute period 
m=2.5 

X : No of arrivals

P(X=0) when M = 2.5

\[P(X=0) = 1 - P(X \geq 1) (Complement)\]
\[= 1 - 0.9179\]
\[= 0.0821\]


%--------------------------------------- %

\subsection{Example}
Thirty-eight students took the test. The X-axis shows various intervals of scores (the interval labeled 35 includes any score from 32.5 to 37.5). The Y-axis shows the number of students scoring in the interval or below the interval.

\textbf{\emph{cumulative frequency distribution}}A  can show either the actual frequencies at or below each interval (as shown here) or the percentage of the scores at or below each interval. The plot can be a histogram as shown here or a polygon.



\section{Poisson Distribution}

% 2 Marks
% m=2 for 60 Minutes  0.2706706
% m=1 for 30 Minutes  0.3678794

%------------------------------------------------------------------------------------%
A researcher takes a random sample of 500 urban residents and finds that
122 have fibre-optic broadband access. Calculate a 90% Confidence Interval for
the true percentage of residents who have fibre-optic broadband access.

The following table gives the results of operations in a hospital according to the complexity of the
operation.


% & & \\ \hline
% Successful & 1990 & 950 \\ \hline
% Unsuccessful & 10 & 50\\ \hline

Let A be the event that an operation is simple and B be the event that an
operation is successful. Calculate P r(B), P r(A|B), P r(A|BC ), P r(B|A) and
P r(B ∩ A). 

%--------------------------------------- %




Past experience shows that there, on average, are 2 traffic accidents on a particular stretch of road every week. 
\\
\bigskip
What is the probability of: 
\begin{itemize}
	
	\item Four accidents during a randomly selected week?  
	
	\item No accidents during a randomly selected week?  
\end{itemize}
%======================================================================================== %
\begin{itemize}
	
	\item The Poisson mean $\lambda$ = 2  per week.
	\item (Unit period is 1 week for both questions)
	\item We use this following formula
	\[ P(X=k) =  \frac{e^{-\lambda} \times \lambda ^k}{k!}  \]
	\item Using our value for the Poisson mean
	\[ P(X=k) =  \frac{e^{-2} \times 2^k}{k!} . \]
	
\end{itemize}

%========================================================================================== %

Probability of four accidents during a randomly selected week : $ P(X = 4)$.

\[ P(X=4) =  \frac{e^{-2} \times 2^4}{4!}  \]



Probability of no accidents during a randomly selected week : $ P(X = 0)$.

\[ P(X=0) =  \frac{e^{-2} \times 2^0}{0!}  \]


%--------------------------------------- %




What is the expected value and standard deviation of the distribution? 

%============================================================================================ %


\end{document}