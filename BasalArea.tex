
\section{Basal area}
This article relies largely or entirely on a single source. Relevant discussion may be found on the talk page. Please help improve this article by introducing citations to additional sources. (August 2018)

Basal area is the area of a given section of land that is occupied by the cross-section of tree trunks and stems at the base. The term is used in forest management and forest ecology.
\subsection{Relationship to mass}

leaf mass (mL) can be estimated from Basal area (BA) (following Whittaker and Marks (1975), based on measurements of eastern North American trees from various spp): mL = 113 000 × BA0.855
Estimation from diameter at breast height

In most countries, this is usually a measurement taken at the diameter at breast height (1.3m or 4.5 ft) of a tree above the ground and includes the complete diameter of every tree, including the bark. Measurements are usually made for a plot and this is then scaled up for 1 hectare of land for comparison purposes to examine a forest's productivity and growth rate.

\begin{itemize}
\item To estimate a tree's basal area B A {\displaystyle BA} BA, use the tree's diameter at breast height D B H {\displaystyle DBH} DBH in inches with the following formula:

B A = π × ( D B H / 2 ) 2 144 {\displaystyle BA={\frac {\pi \times (DBH/2)^{2}}{144}}} BA={\frac {\pi \times (DBH/2)^{2}}{144}}

(Note: The factor of 144 is there to convert from Sq Inches to Sq Feet)

\item This formula simplifies to: B A = 0.005454 × D B H 2 \[{\displaystyle BA=0.005454\times DBH^{2}} \]

The result will be in ft2.

\item For the DBH in cm use: B A = 0.00007854 × D B H 2 \[{\displaystyle BA=0.00007854\times DBH^{2}} \]

The result will be in m2.
\end{itemize}
The basal area of a forest stand can be found by adding the basal areas (as calculated above) of all of the trees in an area and dividing by the area of land in which the trees were measured. Basal area is generally expressed as ft2/acre or m2/ha.

A wedge prism can be used to quickly estimate the basal area per hectare. 
To find basal area using this method, simply multiply your BAF (Basal Area Factor) by the number of "in" trees in your variable radius plot. The BAF will vary based on the prism used, common BAFs include 5/8/10, and all "in" trees are those trees, when viewed through your prism from plot centre, that appear to be in-line with the standing tree on the outside of the prism.
\section{Worked example}

Lets say you carried out a survey using a variable radius plot with angle count sampling (wedge prism) and you selected a Basal Area Factor (BAF) of 4. If your first tree had a diameter at breast height (DBH) of 14cm, then the standard way of calculating how much of 1ha was covered by tree area (scaling up from that tree to the hectare) would be:

\[(BAF/((DBH+0.5)² × π/4))) × 10,000\]

\begin{itemize}
\item BAF, in this case 4, is the BAF selected for the sampling technique.
\item DBH, in this case 14 (this uses an assumed diameter, when actually used is the radius perpendicular to the tangent line)
\item The + 0.5 allows under and over measurement to be accounted for.
\item The π/4 converts the rest to the area.
\end{itemize}

In this case this means in every Ha there is 242m² of tree area according to this sampled tree being taken as representative of all the unmeasured trees.
Fixed area plot

It would also be possible to survey the trees in a Fixed Area Plot (FAP). Also called a Fixed Radius Plot. In the case that this plot was 100m². Then the formula would be

\[(DBH+0.5)²X π/4\]

\subsection*{References}

    R. Hédl, M. Svátek, M. Dancak, Rodzay A.W., M. Salleh A.B., Kamariah A.S. A new technique for inventory of permanent plots in tropical forests: a case study from lowland dipterocarp forest in Kuala Belalong, Brunei Darussalam, In Blumea 54, 2009, p 124–130. Published 30. 10. 2009.

\end{document}
