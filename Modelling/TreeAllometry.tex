\section{Tree Allometry}

Tree allometry establishes quantitative relations between some key characteristic dimensions of trees (usually fairly easy to measure) and other properties (often more difficult to assess). To the extent these statistical relations, established on the basis of detailed measurements on a small sample of typical trees, hold for other individuals, they permit extrapolations and estimations of a host of dendrometric quantities on the basis of a single (or at most a few) measurements.

The study of allometry is extremely important in dealing with measurements and data analysis in the practice of forestry. 

Allometry studies the relative size of organs or parts of organisms. Tree allometry narrows the definition to applications involving measurements of the growth or size of trees. Allometric relationships often are used to estimate difficult tree measurements, such as volume, from an easily measured attribute such as diameter at breast height (DBH).

The use of allometry is widespread in forestry and forest ecology. In order to develop an allometric relationship there must be a strong relationship and an ability to quantify this relationship between the parts of the subject measured and the other quantities of interest.[1] Also when developing this equation one must play in factors which affect tree growth such as age, species, site location, etc.[2] Once all these guidelines are met, one may attempt to develop an allometric equation. 

\end{document}
