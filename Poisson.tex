\documentclass[IntroMain.tex]{subfiles} 
\begin{document}
	%--------------------------------------------------------------------------------------%
	%--------------------------------------------------------------------------------------%
	%---------------------------------------------------------------------------%
\frame{
	\frametitle{The Poisson Probability Distribution}
	\begin{itemize}
		\item A Poisson random variable is the number of successes that result from a Poisson experiment.
		\item The probability distribution of a Poisson random variable is called a Poisson distribution.
		\item Very Important: This distribution describes the number of occurrences in a \textbf{\emph{unit period (or space)}}
		\item Very Important: The expected number of occurrences is $m$
	\end{itemize}
}
\begin{frame}
	\frametitle{The Poisson Probability Distribution}
	We use the following notation.
	\[X \sim Poisson(m) \]
	Note the expected number of occurrences per unit time is conventionally denoted $\lambda$ rather than $m$.
	\bigskip
	As the Murdoch Barnes cumulative Poisson Tables (Table 2) use $m$, so shall we. Recall that Tables 2 gives values of the probability $P(X \geq r )$, when X has a Poisson distribution with
	parameter $m$.
	
\end{frame}
%---------------------------------------------------------------------%
%---------------------------------------------------------------------------%
\begin{frame}
	\frametitle{The Poisson Probability Distribution}
	Consider cars passing a point on a rarely used country road. Is this a Poisson Random Variable?
	Suppose
	\begin{enumerate}
		\item Arrivals occur at an average rate of $m$ cars per unit time.
		\item The probability of an arrival in an interval of length k
		is constant.
		\item The number of arrivals in two non-overlapping
		intervals of time are independent.
	\end{enumerate}
	This would be an appropriate use of the Poisson Distribution.
\end{frame}

%---------------------------------------------------------------------%
\begin{frame}
	\frametitle{Changing the unit time.}
	
	\begin{itemize}
		\item The number of arrivals, X, in an interval of length $t$ has a
		Poisson distribution with parameter $\mu = mt$.
		\item $m$ is the expected number of arrivals in a unit time period.
		\item $\mu$ is the expected number of arrivals in a time period $t$, that is different from the unit time period.
		\item Put simply : if we change the time period in question, we adjust the Poisson mean accordingly.
		\item If 10 occurrences are expected in 1 hour, then 5 are expected in 30 minutes. Likewise, 20 occurrences are expected in 2 hours, and so on.
		\item (Remark : we will not use $\mu$ in this context anymore).
	\end{itemize}
\end{frame}


%---------------------------------------------------------------------%
\begin{frame}
	\frametitle{Poisson Example}
	A motor dealership which specializes in agricultural machinery sells one vehicle every 2 days, on average. Answer the following questions.
	\begin{enumerate}
		\item  What is the probability that the dealership sells at least one vehicle in one particular day?
		\item  What is the probability that the dealership will sell exactly one vehicle in one particular day?
		\item  What is the probability that the dealership will sell 4 vehicles or more in a six day working week?
	\end{enumerate}
\end{frame}

%---------------------------------------------------------------------%
\begin{frame}
	\frametitle{Poisson Example}
	
	\begin{enumerate}
		\item Expected Occurrences per Day: m = 0.5
		\item Probability that the dealership sells at least one vehicle in one particular day?
		\[ P(X \geq 1) = 0.3935 \]
		\item Probability that the dealership will sell exactly one vehicle in one particular day?
		\[ P(X = 1) = P(X \geq 1) - P(X \geq 2)  = 0.3935 - 0.0902 = 0.3031\]
		\item Probability that the dealership will sell 4 vehicles or more in a six day working week?
		\begin{itemize}
			\item For a 6 day week, m=3
			\item $P(X \geq 4) = 0.3528$
		\end{itemize}
	\end{enumerate}
\end{frame}

%---------------------------------------------------------------------%
\begin{frame}
	\frametitle{Knowing which distribution to use}
	\begin{itemize}
		\item For the end of semester examination, you will be required to know when it is appropriate to use the Poisson distribution, and when to use the binomial distribution.
		\item Recall the key parameters of each distribution.
		\item Binomial : number of \textbf{\emph{successes}} in $n$ \textbf{\emph{independent trials}}.
		\item Poisson : number of \textbf{\emph{occurrences}} in a \textbf{\emph{unit space}}.
	\end{itemize}
\end{frame}
%---------------------------------------------------------------------------%
\frame{
\frametitle{Characteristics of a Poisson Experiment}
A Poisson experiment is a statistical experiment that has the following properties:
\begin{itemize}
\item The experiment results in outcomes that can be classified as successes or failures.
\item The average number of successes (m) that occurs in a specified region is known.
\item The probability that a success will occur is proportional to the size of the region.
\item The probability that a success will occur in an extremely small region is virtually zero.
\end{itemize}
Note that the specified region could take many forms. For instance, it could be a length, an area, a volume, a period of time, etc.
}

%---------------------------------------------------------------------------%
\frame{
\frametitle{Poisson Distribution}
A Poisson random variable is the number of successes that result from a Poisson experiment.

The probability distribution of a Poisson random variable is called a Poisson distribution.

Given the mean number of successes ($m$) that occur in a specified region, we can compute the Poisson probability based on the following formula:
}

%---------------------------------------------------------------------------%
\frame{
\frametitle{The Poisson Probability Distribution}
\begin{itemize}
\item The number of occurrences in a unit period (or space)
\item The expected number of occurrences is $m$
\end{itemize}
}

%---------------------------------------------------------------------------%
\frame{
\frametitle{Poisson Formulae}
The probability that there will be $k$ occurrences in a unit time period is denoted $P(X=k)$, and is computed as follows.
\Large
\[ P(X = k)=\frac{m^k e^{-m}}{k!} \]

}
%---------------------------------------------------------------------------%
\frame{
\frametitle{Poisson Formulae}
Given that there is on average 2 occurrences per hour, what is the probability of no occurences in the next hour? \\ i.e. Compute $P(X=0)$ given that $m=2$
\Large
\[ P(X = 0)=\frac{2^0 e^{-2}}{0!} \]
\normalsize
\begin{itemize}
\item $2^0$ = 1
\item $0!$ = 1
\end{itemize}
The equation reduces to
\[ P(X = 0)=e^{-2} = 0.1353\]
}
%---------------------------------------------------------------------------%
\frame{
\frametitle{Poisson Formulae}
What is the probability of one occurrences in the next hour? \\ i.e. Compute $P(X=1)$ given that $m=2$
\Large
\[ P(X = 1)=\frac{2^1 e^{-2}}{1!} \]
\normalsize
\begin{itemize}
\item $2^1$ = 2
\item $1!$ = 1
\end{itemize}
The equation reduces to
\[ P(X = 1) = 2 \times e^{-2} = 0.2706\]
}
\end{document}